\documentclass[a4paper,12pt]{article}
\usepackage[utf8]{inputenc}
\usepackage[T1]{fontenc}
\usepackage[francais]{babel}
\usepackage{graphicx}
\usepackage[left=3cm,right=3cm,top=4cm,bottom=4cm]{geometry}
\pagestyle{plain}

\title{BE de Cyber Physiques}
\author{Grégoire Martini \& Félix Schaller }
\date{23 Mai 2016}

\begin{document}
\maketitle

\bigskip
\bigskip
\bigskip
\tableofcontents
\newpage


\section{TP1 : Modèle continu du pendule simple}
\subparagraph{Modèle}
\subparagraph{Simulation}
\subparagraph{Analyse}
Basique

\clearpage
\section{TP2 : Modèle continu structuré}
\subsection{Pendule simple}
\subparagraph{Modèle}
\subparagraph{Simulation}
\subparagraph{Analyse}
Structuré

\subsection{Pendule inversé avec contrôleur par retour d'état}
\subparagraph{Modèle}
\subparagraph{Simulation}
\subparagraph{Analyse}
Force et perturbation

\clearpage
\section{TP4 : Modèle continu et discret du robot Lego}
\subsection{Modèle continu}
\subparagraph{Modèle}
\subparagraph{Simulation}
\subparagraph{Analyse}
Modélisation, synthèse, estimation

\subsection{Modèle discret}
\subparagraph{Modèle}
\subparagraph{Simulation}
\subparagraph{Analyse}
dito

\clearpage
\section{TP5-6 : Application embarquée sur le robot Lego}
\subparagraph{Modèle}

\subparagraph{Exécution}


\end{document}          

