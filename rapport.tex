\documentclass[a4paper,12pt]{article}
\usepackage[utf8]{inputenc}
\usepackage[T1]{fontenc}
\usepackage[francais]{babel}
\usepackage{graphicx}
\usepackage[left=3cm,right=3cm,top=4cm,bottom=4cm]{geometry}
\pagestyle{plain}

\title{BE de Cyber Physiques}
\author{Grégoire Martini \& Félix Schaller }
\date{23 Mai 2016}

\begin{document}
\maketitle

\bigskip
\bigskip
\bigskip
\tableofcontents
\newpage


\section{TP1 : Modèle continu du pendule simple}
\subparagraph{Modèle}
\subparagraph{Simulation}
\subparagraph{Analyse}
var l V0 alpha0

modifie
l'amplitude (à determiner)
la période (~ inversement proportionel)

\clearpage
\section{TP2 : Modèle continu structuré}
\subsection{Pendule simple}
\subparagraph{Modèle}
\subparagraph{Simulation}
\subparagraph{Analyse}
var l V0 alpha0

verifier obtient même résultats que précédement

\subsection{Pendule inversé avec contrôleur par retour d'état}
\subparagraph{Modèle}
\subparagraph{Simulation}
\subparagraph{Analyse}
a partir de ce point l = 0.07, alpha0 = 0, v0 = 0
perturbation a 20\%, periode 1s et 1 rd/s
var k1, k2

étudier stabilité controle

\clearpage
\section{TP4 : Modèle continu et discret du robot Lego}
\subsection{Modèle continu}
\subparagraph{Modèle}
\subparagraph{Simulation}
\subparagraph{Analyse}

var k1, k2 k3 k4 trouvé par calcul
analyse estimateur

CODE A FAIRE

\subsection{Modèle discret}
\subparagraph{Modèle}
\subparagraph{Simulation}
\subparagraph{Analyse}

dito


\clearpage
\section{TP5-6 : Application embarquée sur le robot Lego}
\subparagraph{Modèle}

\subparagraph{Exécution}


\end{document}          

